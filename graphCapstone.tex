%%%%%%%%%%%%%%%%%%%%%%%
%%% Document Set up %%%
%%%%%%%%%%%%%%%%%%%%%%%

\documentclass[11pt]{amsart}


% you can add extra packages if you need them
\usepackage{amsfonts, amssymb, amscd, amsmath, enumerate, verbatim, calc, thumbpdf, mathrsfs, graphicx, multicol, tabu}


\setlength{\oddsidemargin}{0.25in}  % please do not change
\setlength{\evensidemargin}{0.25in}  % please do not change
\setlength{\marginparwidth}{0in}  % please do not change
\setlength{\marginparsep}{0in}  % please do not change
\setlength{\marginparpush}{0in}  % please do not change
\setlength{\topmargin}{0in}  % please do not change
\setlength{\footskip}{.3in}  % please do not change
\setlength{\textheight}{8.75in}  % please do not change
\setlength{\textwidth}{6in}  % please do not change
\setlength{\parskip}{4pt}  % please do not change


\theoremstyle{plain}  % default
\newtheorem{thm}{Theorem}[section]
\newtheorem{lem}[thm]{Lemma}
\newtheorem{cor}[thm]{Corollary}
\newtheorem{prop}[thm]{Proposition}
\newtheorem{mythm}[thm]{My Great Result}

\theoremstyle{definition}
\newtheorem{defin}[thm]{{Definition}}
\newtheorem{ex}[thm]{Example}

\theoremstyle{remark}
\newtheorem{rem}[thm]{Remark}
\newtheorem*{note}{Note}
\newtheorem{case}{Case}

\numberwithin{equation}{thm}

\newcommand{\dims}{\operatorname{dims}}
\newcommand{\rank}{\operatorname{rank}}

\DeclareMathOperator{\lcm}{lcm}



%%%%%%%%%%%%%%%%%%%%%%%%%%%%%%%%%
%%% Document Meta Information %%%
%%%%%%%%%%%%%%%%%%%%%%%%%%%%%%%%%


\begin{document}


\title[Abstract Algebra Project]{Investigation of resolving sets and metric dimension}
\thanks{Dogan Comez}
\author[K.~ Ryan]{Kyle Ryan\\{Advisors: Trevor McGuire, Warren Shreve}}


\address{Department of Mathematics 2750\\ North Dakota State University\\PO BOX 6050\\ Fargo, ND 58108-6050\\ USA}

\email{kyleryanmn@gmail.com, trevor.mcguire@ndsu.edu, warren.shreve@ndsu.edu}





\maketitle
% \setcounter{page}{***} % Please do not change this line



%%%%%%%%%%%%%%%%%%%%
%%% Introduction %%%
%%%%%%%%%%%%%%%%%%%%
1. introduction
\begin{abstract}
        For any graph G it is possible to describe each of its vertices uniquely with respect to an ordered subset of vertices of G called a resolving set.
    In this project, we investigate properties of associated with minimal resolving sets, known as bases and conditions under which changing G affects its bases. 
\end{abstract}
    The basics of what you're doing: relevant tools, defs techniques etc
 \begin{defin}[Representation of vertex(with respect to W)]:\\
    For an ordered subset $W=(w_1, w_2, ..., w_i, w_{i+1}, ...)$ of vertices in $G$, the representation of a vertex $v \in G$ with respect to $W$ denoted $rep(v/W)$ is: \\
  $rep(v/W)=(d(v,w_1), d(v,w_2), ..., d(v,w_i), d(v,w_{i+1}), ... )$
 \end{defin}
    
 \begin{defin}[Resolving Set]:\\
  
 \end{defin}
 
 \begin{defin}[Basis of a graph]:\\
  
 \end{defin}
 
 \begin{defin}[Metric Dimension]:\\
  
 \end{defin}

 \begin{defin}
  what is known, what is not known
 \end{defin}


%%%%%%%%%%%%%%%%%%%%%%%%%%%%%%%%
%%% References and Citations %%%
%%%%%%%%%%%%%%%%%%%%%%%%%%%%%%%%
\cite{Chartrand:2010:GDF:1941879}
\nocite{*}
\bibliographystyle{plain}
\bibliography{mybibj;l}  % add references to the list of references saved in the file mybib.bib


\end{document}
